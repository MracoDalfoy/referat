\section{Методы защиты информации от вирусов}

\subsection{Профилактические мероприятия}

Существует несколько способов, с помощью которых мы можем защитить нашу систему и данные от вирусного воздействия: 

1.	 \textbf{Поддерживайте свое программное обеспечение в актуальном состоянии}

 Компании-разработчики программного обеспечения, такие как Microsoft и Oracle, регулярно обновляют свое программное обеспечение для устранения ошибок. Неактуальность версии программного обеспечения может быть использовано вирусом для возможности заражения компьютера.

 2.	\textbf{Не переходите по ссылкам в электронных письмах} 

Хорошее эмпирическое правило заключается в том, что если вы не узнаете отправителя электронного письма, то не нажимайте ни на какие ссылки в письме.

3.	\textbf{Используйте бесплатное или платное антивирусное программное обеспечение} 

Вам не нужно покупать программное обеспечение для защиты вашего компьютера или годовую подписку, чтобы позаботиться о новейшей защите от вирусов. Для пользователей Windows Microsoft Security Essentials бесплатна. Avast — это еще одна бесплатная антивирусная программа.

4.	\textbf{Резервная копия данных} 

Если у вас нет защиты системы, то вы должны периодически делать резервную копию своих данных. Три основных варианта резервного копирования: Внешний накопитель, Онлайн-служба резервного копирования, Облачное хранилище. Используйте такие услуги, как Google drive, Google docs для хранения файлов. Некоторые облачные хранилища имеют определенный бесплатный объем хранения данных. Виртуальное хранилище — это ресурс для сохранения ваших данных.

5.	\textbf{Пароль}


В то время как некоторые люди используют эквивалентный пароль для всего, избегайте этой практики. Длина пароля должна составлять не менее восьми символов. Надежный пароль, состоит из букв, цифр и символов. Периодически меняйте свой пароль.

6.	\textbf{Брандмауэр}

Если в вашей системе запущено антивирусное программное обеспечение, то это не значит, что у вас есть брандмауэр. Компьютеры на базе ОС Windows и IOS имеют встроенное программное обеспечение брандмауэра. Убедитесь, что он включен.

7.	\textbf{Блокировщик всплывающих окон} 

Веб-браузеры имеют возможность предотвращать всплывание окон. Вы можете использовать Add Blocker, чтобы заблокировать вредоносную и навязчивую рекламу на веб-сайтах.

\subsection{Антивирусное программное обеспечение}

Антивирусное программное обеспечение обнаруживает и удаляет вирусы и другие вредоносные программы, такие как черви, трояны, рекламное ПО и многое другое. Это программное обеспечение предназначено для использования в качестве превентивного подхода к кибербезопасности, чтобы остановить угрозы до того, как они попадут на ваш компьютер и вызовут проблемы.

Антивирусное программное обеспечение работает, сканируя входящие файлы или код, который передается через сетевой трафик. Компании, которые создают это программное обеспечение, составляют обширную базу данных уже известных вирусов и вредоносных программ и регулярно обновляют ее. Когда файлы, программы и приложения передаются на ваш компьютер, антивирус сравнивает их со своей базой данных, чтобы найти совпадения. Совпадения, похожие или идентичные базе данных, изолируются в карантин, сканируются и удаляются. В режиме реального времени, когда вы просматриваете веб-страницы, отправляете электронные письма, смотрите потоковое видео или делаете что-либо еще в Интернете, программное обеспечение предупредит вас, чтобы вы не нажимали на какие-либо веб-сайты или файлы, которые могут представлять угрозу вашей безопасности в Интернете.

По данным исследовательского портала US News 360 Reviews на 4 апреля 2024 год составлен рейтинг популярных антивирусных программ для Windows 11 и Windows 10:

1. Bitdefender. Цена от 40 долларов и выше. Присутствует бесплатная пробная версия.

2. AVG Free Antivirus. Бесплатный. 

3. Malwarebytes Премиум. Цена от 40 долларов и выше. Присутствует бесплатная пробная версия.

4. Norton 360 Select с Lifelock Select. Цена от 100 долларов за 10 устройств и выше. Бесплатная версия отсутствует.

5. Eset Mobile Security Premium. Цена от 13 долларов и выше. Присутствует базовая бесплатная версия.

